\documentclass[english]{article}
\usepackage[latin1]{inputenc}
\usepackage{babel}
\usepackage{verbatim}

%% do we have the `hyperref package?
\IfFileExists{hyperref.sty}{
   \usepackage[bookmarksopen,bookmarksnumbered]{hyperref}
}{}

%% do we have the `fancyhdr' package?
\IfFileExists{fancyhdr.sty}{
\usepackage[fancyhdr]{latex2man}
}{
%% do we have the `fancyheadings' package?
\IfFileExists{fancyheadings.sty}{
\usepackage[fancy]{latex2man}
}{
\usepackage[nofancy]{latex2man}
\message{no fancyhdr or fancyheadings package present, discard it}
}}

\setDate{2016/09/26}    %%%% must be manually set, if rcsinfo is not present
\setVersionWord{Version:}  %%% that's the default, no need to set it.
\setVersion{0.4}

\begin{document}

\begin{Name}{1}{git-lint}{Elf M. Sternberg}{Utilities}{Git Lint - A smart lint wrapper around your git repository}
  \Prog{Git Lint} - A smart wrapper around your git repository
\end{Name}

\section{Synopsis}
%%%%%%%%%%%%%%%%%%

\Prog{git lint} [<options>..] [filenames..]

\section{Description}
%%%%%%%%%%%%%%%%%%%%%
\Prog{git-lint} provides a simple, single call to perform syntatic and stylistic
checks of your repository, in order to ensure they comply with your standards
before you commit your work.

\section{Options}
%%%%%%%%%%%%%%%%%
\begin{Description}

\item[\OptArg{-o}{ linters}]Specify the list of linters to run, excluding all others
\item[\OptArg{-x}{ linters}]Specify the list of linters to exclude, running all others
\item[\Opt{-l}]List configured linters.  The list will specify whether or not the
  command line argument provided leads to an identifiable executable linter.
\item[\Opt{-b}]Run check from the repository base, rather than the current working directory.
\item[\Opt{-a}]Check all files, not just those that have changed.
\item[\Opt{-e}]Check everything (short for {-a -b}).
\item[\Opt{-w}]Check the workspace (default when run as \Prog{git-lint}).
\item[\Opt{-s}]Check the staging area.  \Prog{git-lint} stashes the
  current workspace, restoring it to the same state as the index.
  After the check, \Prog{git-lint} restores the workspace and attempts
  to reset all fill times correctly.

  
\item[\Opt{-g}]Only error if lint failures overlap diffed regions.
\item[\Opt{-p}]Error if a lint failure happens anywhere in a checked file.
\item[\Opt{-t}]Group output by linter first, then filenames [default].
\item[\Opt{-f}]Group output by filenames first, then linter.
\item[\Opt{-d}]Dryrun - Report what \Prog{git-lint} would do, but don't actually run linters.
\item[\OptArg{-c}{ config-file}]Specify an alternative configuration file.
\item[\Opt{-h}]Print short help message
\item[\Opt{-v}]Print version information

\end{Description}

\section{Requirements}
%%%%%%%%%%%%%%%%%%%%%%

\begin{description}\setlength{\itemsep}{0cm}
\item[An IBM or Lenovo Thinkpad with HDAPS] \Prog{thinksaber} only
  runs on laptops with accelerometers, which get their values through
  the HDAPS joystick emulator.

\item[PyGame] \Prog{thinksaber} uses the PyGame library
  (www.pygame.org), which in turn has dependencies on the Simple
  Direct Layer gaming library as well as Python.  Most Linux
  distributions either come with this stock or provide it through the
  installation tool.  Pygame is a dependency of a number of popular
  Linux games, so if you have any games on your system it's entirely
  likely this has already been done for you.

\item[Make] If you want to install the system with the distributed
     \File{Makefile}, you need GNU-\Prog{make}. If you don't have it, you
     should execute the steps shown in the \File{Makefile} manually.

\end{description}

\section{Acknowledgements}
%%%%%%%%%%%%%%%%%%%%%%

\Prog{Thinksaber} is obviously inspired by the program MacSaber, and I'm
grateful to the MacSaber people for assembling the Star Wars sound
effects collection needed to make it so successful.

\Prog{Thinksaber} uses a motion-detection algorithm derived from the
one written by Tatsuhiko Miyagawa (miyagawa at gmail.com) for his own
\Prog{thinkpad-saber} program, which ran only under Perl for Windows.
Obviously, I think mine's better.

\section{Changes}
%@% IF LATEX %@%
{\small\verbatiminput{CHANGES}}
%@% ELSE %@%
Please check the file \URL{CHANGES} for the list of changes.
%@% END-IF %@%

\section{Version}
%%%%%%%%%%%%%%%%%

Version: \Version\ of \Date.

\section{License and Copyright}
%%%%%%%%%%%%%%%%%%%%%%%%%%%%%%%

\begin{description}
\item[Copyright] \copyright\ 2008, Elf M. Sternberg,
     \Email{Elf.Sternberg@gmail.com}

\item[License] This program can be redistributed and/or modified under
  the terms of the GNU Public License, version 2.  You should have
  found a copy of this licence with this distribution in the file
  \File{COPYING}.

\end{description}

\section{Author}
%%%%%%%%%%%%%%%%

\noindent
Elf M. Sternberg
Email: \Email{Elf.Sternberg@gmail.com}  \\
WWW: \URL{http://www.elfsternberg.com}.

\LatexManEnd

\end{document}
